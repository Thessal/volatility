\documentclass{article}
\begin{document}

\section{Notation}
\begin{itemize}
    \item 1-second last mid price was observed for 1-month 
    \item Daily volatility is being estimated
    \item Sampling period : K (60 * 60 * 24 = 86400)
    \item Sequence length : n (86400 * 30 = 2592000)
    \item time : $0<t<n$
    \item log-price : $x_t$ 
    \item log-return : $r_t$ 
\end{itemize}

\section{Jump adjust}
Removes jumps and stitch the log-pric

\subsection{Truncation}
Sub-Gaussian concentration condition was used to determine threshold parameter $\tau$ for each day.
$$\tau = \sigma \sqrt{K} \sim \sqrt{\sum_t^{t+K}{r_t^2}}$$
Truncation function with parameter $\tau$ was applied to the log return.
$$\Psi_\tau(x) = \mathrm{sgn}(x) \min(|x|, \tau)$$

\subsection{Truncation with MLE}
Huber loss of truncated series was minimized with L1 regularization, using daily volatility as prior.

\begin{itemize}
    \item Daily volatility : $\sigma^2_t =  \frac{1}{n-K}\sum_{t=1}^{n-K} \left(\frac{1}{K} \sum_{i=t}^{t+K} r_i \right)^2 $
    \item Huber loss : $l_{\tau'}(x) = \mathrm{sgn}(x) \min(\frac{x^2}{2}, {\tau'} |x| - \frac{{\tau'}^2}{2})$\\
         with $\tau' = \sigma \sqrt{K} \sim \sqrt{\sum_t^{t+K}{r_t^2}}$ 
    \item Truncation : $\Psi_\tau(x) = \mathrm{sgn}(x) \min(|x|, \tau)$
    \item Optimal truncation threshold $\hat\tau = \arg\min \left( \sum_{t}^{t_K}\tau l_{\tau'}(\sigma_t-\Psi(r_t)) \right)$
\end{itemize}

\subsection{Benchmark}
Does not remove jump noise for benchmark purpose

\section{Microstructure noise adjust}
Removes high-frequency component

\subsection{Pre-Averaging}
Hourly rolling average 

\subsection{Fracdiff}
Ensures stationarity while minimizng information loss.

\subsection{Pad\'e transform}
Models jumps as Lorentz peak of oscillator resonance.

\section{Volatility estimation}
Estimate volatility, for given raw log-price and denoised log-price.
\subsection{RV}
\subsection{PRV}
\subsection{TRV}

\section{Evaluation}
Compare methods based on information measure
\subsection{KLD of Conditional distribution}
Measures the amount of information in time Filtration $\mathcal{F}_t$.
$$\mathrm{KL} ( P | P_0 ) $$
where $P = P(\sigma_t|\sigma_t{-1}) = P(\sigma_{t}|\mathcal{F}_t)P(\sigma_{t-1})$ and $P_0 = P(\sigma_t)P(\sigma_t) = P(\sigma_{t})P(\sigma_{t-1})$.
P is estimated using kernel density estimation.

\subsection{Self-similarity dimension}.
Fractal dimension is used to measure information amount, using box-counting method.

\subsection{Minkowski Measure}
For given fractal dimension, measures persistency.

\end{document}